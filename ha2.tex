%       File: main.tex
%     Created: Tue Oct 16 06:00 PM 2018 CST
% Last Change: Tue Oct 16 06:00 PM 2018 CST
%
\documentclass{article}

\usepackage[left=3cm, right=3cm]{geometry}
\usepackage[utf8]{inputenc}
\usepackage{amsmath}
\usepackage{mathtools}
\usepackage{graphicx}
\usepackage{todonotes}

\definecolor{forestgreen}{RGB}{11,102,35}
\usepackage{fancyhdr}
\pagestyle{fancy}

\title{Home Assignment 2}
\author{André Hedesand}
\date{October 2018}

\rhead{André Hedesand, 970318-7832}

\begin{document}
\maketitle
\thispagestyle{fancy}
\section*{Solutions}
\begin{enumerate}
    \item % 1
        \emph{``A rotor on a helicopter has four blades and rotates with 4 500 revolutions/minute, counter
            clockwise, (note, revolutions/minute). The rotor is filmed with a digital camera that takes 50
            pictures/second. What will the perceived rotation speed be when watching the recorded film sequence?
            What will be the resulting direction of rotation? How does it look if the rotor goes slightly 
            slower, with 4 499 revolutions/minute?''} 

        Let the frequency of the helicopters' rotor be $4500rpm = \frac{4500}{60}Hz = 75Hz$. 
        The video camera does the sampling with $F_s=50Hz$. This means we get the relative 
        frequency $f=\frac{75Hz}{50Hz}=1.5=0.5+1$.

        Since we are under-sampling, the video of the rotor is only going to show $f=0.5$(half 
        a turn). When really the rotors are spinning 1.5 turns per picture. 

        If we then can't distinguish between the four rotor blades, it will look like the     
        rotorblades are static in place.
        \\

        Lets now assume the helicopters' rotor is spinning at $4499rpm=74.98Hz$. With 
        this we get $f=1.499=0.499+1$. Now the video camera shows $0.499$ revolutions per     
        minute. This will give the effect of a slow movement \emph{clockwise}(The rotor is 
        actually spinning \emph{counter clockwise}).
        \\

        We could also take an extra look at the rotor: Because the four rotor blades are seated in a $+$-sign,
        only rotations less than $\frac{\pi}{2}rad$ can be distinguished. This means the $f$ in the first case 
        can be written as $f = 1.5 = 0 + 0.25 * 6$ which means that the rotor blades appears to stand still as 
        we've seen before. The second case can be written as $f = 1.499 = (-0.001) + 0.25 * 6$ where the blade 
        appears to have a relative frequency of $-0.001$.


    \item % 2
        \emph{``The following discrete-time signals are given:''}
        \begin{displaymath}
            x_1(n) = 
            \begin{bmatrix}
                1 & \underline{-3} & -1 & 3
            \end{bmatrix}
            ,
            x_2(n) = 
            \begin{bmatrix}
                \underline{-1} & -1 & 2 & -2 & 1 & -2
            \end{bmatrix}
        \end{displaymath}

        \emph{``Determine the following: (3 out of 4 correct answers gives full points.)''}
        \begin{enumerate}
            \item % 2a 
                \emph{``The linear convolution of the sequences, that is, $y(n) = x_1(n) * x_2(n)$.''} 
                \\
                \begin{displaymath}
                    y(n)=x_1(n)*x_2(n)=
                    \begin{bmatrix}
                        1 & \underline{-3} & -1 & 3
                    \end{bmatrix}
                    *
                    \begin{bmatrix}
                        \underline{-1} & -1 & 2 & -2 & 1 & -2
                    \end{bmatrix}
                \end{displaymath}

                \begin{displaymath}
                    =
                    \begin{tabular}{c | c c c c}
                        $x_1 * x_2 $ & 1 & \underline{-3} & -1 & 3 \\
                        \hline
                        \underline{-1} & \color{red}-1 & \color{blue}3 & \color{forestgreen}1 & \color{red}-3 \\
                        -1             & \color{blue}-1 & \color{forestgreen}3 & \color{red}1 & \color{blue}-3 \\
                        2 & \color{forestgreen}2 & \color{red}-6 & \color{blue}-2 & \color{blue}6 \\
                        -2 & \color{red}-2 & \color{blue}6 & \color{forestgreen}2 & \color{forestgreen}-6 \\
                        1            & \color{blue}1 & \color{forestgreen}-3 & \color{red}-1 & \color{blue}3 \\
                        -2 & \color{forestgreen}-2 & \color{red}6 & \color{blue}2 & \color{forestgreen}-6
                    \end{tabular}
                    \Longrightarrow
                    y(n)=
                    \begin{bmatrix}
                        \underline{-1} & 2 & 6 & -10 & 2 & 3 & -1 & 5 & -6
                    \end{bmatrix}
                \end{displaymath}
            \item % 2b
                \emph{``The circular convolution modulo 4, that is, $y(n) = x1(n) *_4 x2(n)$.''}
                \\
                \begin{displaymath}
                    x_1(n)*x_2(n) =
                    \begin{tabular}{c|ccccccccc}
                        $x_1 *_4 x_2$   & \color{gray}3 & \color{gray}1 & \color{gray}-3  & \color{gray}-1  &   
                        \color{gray}3   &   1   &   \underline{-3}  &   -1  &   3   \\
                        \hline
                        \underline{-1} & 
                        \color{gray}-3 & \color{gray}-1 & \color{gray}3 & \color{gray}1 & \color{gray}-3 &
                        \color{red}-1 & \color{blue}3 & \color{forestgreen}1 & \color{red}-3 \\
                        -1 & 
                        \color{gray}-3 & \color{gray}-1 & \color{gray}3 & \color{gray}1 & \color{red}-3 & 
                        \color{blue}-1 & \color{forestgreen}3 & \color{red}1 & -3 \\
                        2 & 
                        \color{gray}6 &\color{gray}2 & \color{gray}-6 & \color{red}-2 & \color{blue}6 & 
                        \color{forestgreen}2 & \color{red}-6 & -2 & 6 \\
                        -2 & 
                        \color{gray}-6 & \color{gray}-2 & \color{red}6 & \color{blue}2 & \color{forestgreen}-6 & 
                        \color{red}-2 & 6 & 2 & -6 \\
                        1 & 
                        \color{gray}3 & \color{red}1 & \color{blue}-3 & \color{forestgreen}-1 & \color{red}3 & 
                        1 & -3 & -1 & 3 \\
                        -2 & 
                        \color{red}-6 & \color{blue}-2 & \color{forestgreen}6 & \color{red}2 & \color{gray}-6 & 
                        -2 & 6 & 2 & -6\\
                    \end{tabular}
                    \Longrightarrow
                    y(n)=
                    \begin{bmatrix}
                        \color{red}-5 & \underline{\color{blue}5} & \color{forestgreen}5 & \color{red}-5
                    \end{bmatrix}
                \end{displaymath}
            \item % 2c
                \emph{``The linear correlation of the sequences, that is, $r_{x_1x_2}(n)=x_1(n)*x_2(-n)$.''}
                \\
                \begin{displaymath}
                    x_1(n) = 
                    \begin{bmatrix}
                        1 & \underline{-3} & -1 & 3
                    \end{bmatrix}
                    ,
                    x_2(-n) = 
                    \begin{bmatrix}
                       -2&1 & -2 & 2 & -1 & \underline{-1} 
                    \end{bmatrix}
                \end{displaymath}

                \begin{displaymath}
                    x_1(n)*x_2(-n) =
                    \begin{tabular}{c|cccc}
                        $r_{x_1x_2}$    &   1   &   \underline{-3}  &   -1  &   3   \\
                        \hline
                        -2              &   -2  &   6               &   2   &   -6  \\  
                        1               &   1   &   -3              &   -1  &   3   \\
                        -2              &   -2  &   6               &   2   &   -6  \\  
                        2               &   2   &   -6              &   -2  &   6   \\
                        -1              &   -1  &   3               &   1   &   -3  \\
                        \underline{-1}  &   -1  &   3               &   1   &   -3  \\
                    \end{tabular}
                    \Longrightarrow
                    y(n)=
                    \begin{bmatrix}
                        -2 & \underline{7} & -3 & 1 & -2 & -6 & 10 & -2 & -3 
                    \end{bmatrix}
                \end{displaymath}

            \item % 2d
                \emph{``The circular modulo 5 correlation of the sequences, that is, $r_{x_1x_2}=x_1(n)*_5x_2(-n)$''}
                \\

                We will begin with padding the shorter signal to the length 5:
                \begin{displaymath}
                    x_1(n)=
                    \begin{bmatrix}
                        1 & \underline{-3} & -1 & 3
                    \end{bmatrix}
                    =
                    \begin{bmatrix}
                        1 & \underline{-3} & -1 & 3 & 0
                    \end{bmatrix}
                \end{displaymath}
                \begin{displaymath}
                    x_1(n)*_5x_2(-n) =
                    \begin{tabular}{c|ccccc ccccc}
                        $x_1 *_5 x_2$ & 
                        \color{gray}1 & \color{gray}-3  & \color{gray}-1  & \color{gray}3 & \color{gray}0 &
                        1   &   \underline{-3}  &   -1  &   3  &  0 \\
                        \hline
                        -2 & 
                        \color{gray}-2 & \color{gray}6 & \color{gray}2 & \color{gray}-6 & \color{gray}0 &
                        \color{red}-2 & \color{blue}6 & \color{forestgreen}2 & \color{red}-6 & \color{blue}0 \\
                        1 & 
                        \color{gray}1 & \color{gray}-3 & \color{gray}-1 & \color{gray}3 & \color{red}0 & 
                        \color{blue}1 & \color{forestgreen}-3 & \color{red}-1 & \color{blue}3 & 0 \\ 
                        -2 & 
                        \color{gray}-2 & \color{gray}6 & \color{gray}2 & \color{red}-6 & \color{blue}0 &
                        \color{forestgreen}-2 & \color{red}6 & \color{blue}2 & -6 & 0 \\
                        2 & 
                        \color{gray}2 & \color{gray}-6 & \color{red}-2 & \color{blue}6 & \color{forestgreen}0 &
                        \color{red}2 & \color{blue}-6 & -2 & 6 & 0 \\
                        -1 & 
                        \color{gray}-1 & \color{red}3 & \color{blue}1 & \color{forestgreen}-3 & \color{red}0 &
                        \color{blue}-1 & 3 & 1 & -3 & 0 \\
                        \underline{-1} & 
                        \color{red}-1 & \color{blue}3 & \color{forestgreen}1 & \color{red}-3 & \color{blue}0 &
                        -1 & 3 & 1 & -3 & 0 \\
                    \end{tabular}
                    \Longrightarrow
                \end{displaymath}
                \begin{displaymath}
                    y(n)=
                    \begin{bmatrix}
                        \color{red}-8 & \underline{\color{blue}17} & \color{forestgreen}-5 & \color{red}-2 &\color{blue} -2
                    \end{bmatrix}
                \end{displaymath}
               
                
        \end{enumerate}

    \item % 3
        \emph{``Signals are sampled, down sampled or up sampled and reconstructed ideally according to the items below. Determine what the resulting signal will be.''}
        \begin{enumerate}
            \item % 3a
                \emph{``The signal $cos(2\pi450t)$ is sampled using $F_s = 1000Hz$, and then down sampled by a
                        factor 3 (that is, only every third sample value is kept). The resulting signal is then
                        ideally reconstructed (using $F_s = 1000 Hz$).''}
                \\
                \begin{displaymath}
                    cos(2\pi*450t) \xrightarrow[]{Sampling \textbf{1kHz}} cos(2\pi*\frac{450}{1000}n) = 
                    \left\{ n=3n \right\} cos(2\pi*\frac{1350}{1000}n) = 
                    cos(2\pi* ( \frac{350}{1000} + k )n)
                \end{displaymath}
                \begin{displaymath}
                    \xrightarrow[]{Reconstruction \textbf{1kHz}} cos(2\pi*350t) 
                \end{displaymath}

            \item % 3b
                \emph{``The signal $cos(2\pi1680t)$ is sampled with $F_s = 600 Hz$, up-sampled (that is, after
                        every sample value two zeroes are inserted), and then ideally reconstructed with a
                        new sample rate, $Fs = 500 Hz$''}
                \\

                \begin{displaymath}
                    cos(2\pi*1680t) \xrightarrow[]{Sampling \textbf{600Hz}} cos(2\pi*\frac{1680}{600}n)
                    =
                    cos(2\pi*\frac{28}{10}n)
                    =
                    cos(2\pi*(\frac{8}{10}+k)n)
                    =
                \end{displaymath}
                \begin{displaymath}
                    \left\{ f=\frac{f}{3} \right\} cos(2\pi*(\frac{8}{30}+k)n)
                    \xrightarrow[]{Reconstruction \textbf{500Hz}} cos(2\pi*\frac{400}{3}t)
                \end{displaymath}
        \end{enumerate}
    \end{enumerate}
\end{document}


