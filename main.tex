\documentclass{article}
\usepackage[left=2cm, right=2cm, top=2cm]{geometry}
\usepackage[utf8]{inputenc}

\title{Home Assignment 1}
\author{André Hedesand}
\date{October 2018}

\begin{document}

\maketitle

\section*{Solutions}

\begin{enumerate}
    \item
        \begin{enumerate}
            \item % 1a
                \emph{"Recursive systems are never stable"}
                \\
                \textbf{False.} 
                Recursive systems are IIR systems e.g. 
                $$ y(n) = \frac{1}{2}y(n-1) + x(n) $$
                If we by stability mean BIBO-stability we can simply prove that the system above is stable. The Z-transorm of the system above is
                $$ 
                    Y(z) = \frac{1}{2}z^{-1}Y(z) + X(z) \Rightarrow
                    Y(z)(1 - \frac{1}{2}z^{-1}) = X(z) \Rightarrow
                $$
                $$
                    Y(z) = \frac{1}{1 - \frac{1}{2}z^{-1}}X(z) = 
                    H(z) X(z) \Rightarrow 
                    H(z) = \frac{1}{1 - \frac{1}{2}z^{-1}} \Rightarrow
                $$
                $$
                    h(n) = (\frac{1}{2})^{n} u(n)
                $$
                
                By definition the system above is stable if
                $ \sum_{n=0}^{\infty} |(\frac{1}{2})^{n}| < \infty $.
                This is true because $(\frac{1}{2})^{n}$ goes to zero when $n$ goes to $\infty$.
            
            \item % 1b
                \emph{"A convolution of two sequences in the time domain corresponds to a multiplication of the Z transforms of the signals."}
                \\
                \textbf{True.} This is true by definition. Skriv in formel h'r
            
            \item % 1c
                \emph{"An FIR filter is always stable."}
                \\
                \textbf{True.}
                A FIR filter is always \emph{BIBO stable} and can be discribed with a difference equation:
                $$ y(n) = \sum_{k=0}^{N} a_k x(n-k)$$
                The system is stable as all and every $a_n < \infty$.
                One exception to this is if we allow $N$ to be infinitely large. Then the system isn't stable as the sum goes to infinity.
                
            \item % 1d
                \emph{"A first order IIR filter is stable iff the absolute value of the value of the factor in front of $y(n-1)$ is greater than 1."}
                \\
                \textbf{False.} The opposite is true. We can prove this by contradiction: If we assume the statement above is true we get:
                
                $$
                    y(n) = Ay(n-1) + x(n), A > 1
                $$
                \begin{center}
                    \textit{and}
                \end{center}
                $$
                   \sum_k |h(k)| < \infty
                $$
                
                Lets do the same steps as in \emph{1a)}:
                $$
                   y(n) = Ay(n-1) + x(n) \Rightarrow h(n) = A^nu(n)
                $$
                At last we get
                $$
                  \sum_k |h(k)| < \infty \Rightarrow \sum_{n=0}^{\infty}|A^n| < \infty  (A > 1)
                $$
                Which can't be true as the sum $\sum_{n=0}^{\infty}|A^n| \rightarrow \infty$.
                
            \item % 1e
                \emph{"An IIR-filter is never a linear-phase system."}
                % TODO
                INTE KLAAAAAAAAAAR
                
                
        \end{enumerate} % 1abcde
    \item
        \emph{"A discrete-time system is described by the difference equation:"}
        $$
            y(n)-y(n-1) + \frac{2}{9}y(n-2) = x(n)
        $$
        \begin{enumerate}
            \item % 2a
                \emph{"Determine the system function $H(z)$ and the impulse response $h(n)$ for the system and draw a pole-zero plot. Is the system stable?"}
                
                $$
                    y(n)-y(n-1) + \frac{2}{9}y(n-2) = x(n) \longmapsto 
                    Y(z) - Y(z)z^{-1} + \frac{2}{9} Y(z)z^{-2} = X(z) \Rightarrow
                $$$$
                    Y(z)(1 - z^{1} - \frac{2}{9} z^{-2}) = X(z) \Rightarrow
                    Y(z) = \frac{1}{1 - z^{-1} - \frac{2}{9}z^{-2}} X(z) = H(z)X(z)
                $$
                
                Now we have the system function. To get the impulse response we just need to inverse transform the system function.
                $$
                    H(z) = \frac{1}{1 - z^{-1} - \frac{2}{9}z^{-2}} = 
                    \frac{z^2}{z^2 - z - \frac{2}{9}}
                $$
                $$
                    z^2 - z - \frac{2}{9} = 0 \Rightarrow 
                    z = \frac{1}{2} \pm \sqrt{(\frac{1}{2})^2 + \frac{2}{9}} = 
                    \frac{1}{2} \pm \frac{\sqrt{17}}{6}
                $$
                INTE KLART H"RRR
                
            \item % 2b
                \emph{"The following signal is the input signal to the system"}
                $$
                    x(n) = 3(\frac{1}{2})^nu(n)
                $$
        \end{enumerate}
    \item % 3
        \emph{"The figures below show four pole-zero plots and four impulse responses."}
        \begin{enumerate}
            \item % 3a
                \emph{"Pair the correct plot A, B, C, D with the corresponding impulse response 1, 2, 3, 4."}
                \\
                \begin{center}
                    \begin{tabular}{c|c}
                        A & 1 \\
                        B & 3 \\
                        C & 4 \\
                        D & 2 \\
                    \end{tabular}
                \end{center}
            \item % 3b
                \emph{"Pair the correct plot A, B, C, D with the corresponding statement I, II, III, IV."}
                \\
                \begin{center}
                    \begin{tabular}{c|c}
                        A & I \\
                        B & III \\
                        C & II \\
                        D & IV \\
                    \end{tabular}
                \end{center}
                
                
        \end{enumerate} % 3ab
        
\end{enumerate} % 1,2,3

\end{document}
